\documentclass{resume}
\usepackage{zh_CN-Adobefonts_external} 
\usepackage{linespacing_fix}
\usepackage{cite}
\usepackage{titling}
\usepackage{hyperref}
\usepackage{wasysym}
\hypersetup{
    colorlinks=true,
    linkcolor=cyan,
    filecolor=magenta,
    urlcolor=blue,
}

\author{庄宇林}
\date{\today}  % Update date set to last compile:


\begin{document}
\pagenumbering{gobble}

%***"%"后面的所有内容是注释而非代码,不会输出到最后的PDF中
%***使用本模板,只需要参照输出的PDF,在本文档的相应位置做简单替换即可
%***修改之后,输出更新后的PDF,只需要点击Overleaf中的“Recompile”按钮即可

%在大括号内填写其他信息,最多填写4个,但是如果选择不填信息,
%那么大括号必须空着不写,而不能删除大括号。
%\otherInfo后面的四个大括号里的所有信息都会在一行输出
%如果想要写两行,那就用两次这个指令(\otherInfo{}{}{}{})即可


%***********个人信息**************
\MyName{庄宇林}
\sepspace
\SimpleEntry{软件开发工程师}
\vspace{0.2em}
\SimpleEntry{
    \begin{tabular}{rp{.75\linewidth}}
        \baselineskip=20pt
        \email{} : & \href{mailto:307256078@qq.com}{307256078@qq.com}\\
        \phone{} : & \href{tel:18650395493}{18650395493}
    \end{tabular}
}
\SimpleEntry{{\footnotesize\color{gray} (Last updated \thedate.)}}

%************照片**************
%照片需要放到images文件夹下,名字必须是you.jpg,注意.jpg后缀(可以去resume.cls第101行处修改),如果不需要照片可以不添加此行命令
%0.15的意思是,照片的宽度是页面宽度的0.15倍,调整大小,避免遮挡文字
\yourphoto{0.13}

%***********教育背景**************
\section{求学经历}
    %***第一个大括号里的内容向左对齐,第二个大括号里的内容向右对齐
    %***\textbf{}括号里的字是粗体,\textit{}括号里的字是斜体
    \datedsubsection{\textit{硕士},控制理论与控制工程,\fzu}{2016.9 - 2019.4}
    \datedsubsection{\textit{本科},电气工程与自动化,\fzu}{2012.9 - 2016.7}
\sepspace
% \begin{minipage}[t]{.15\linewidth}
% \hfill \textsc{2016.9 - 2019.4}
% \end{minipage}
% \hfill\vline\hfill
% \begin{minipage}[t]{.80\linewidth}
% {\bf \textit{硕士},控制理论与控制工程}\\{\Fzu} we learning DSP automatics
% \end{minipage}

%************基本**************
\section{个人技能}
    \datedsubsection{\textbf{计算机方面}:}{}
        \begin{itemize}[parsep=0.5ex]
            \item  熟悉Linux操作系统下C开发、内核驱动开发、用户态驱动开发
            \item  熟悉DPDK,OVS与智能网卡驱动
            \item  熟悉基于DPDK/OVS实现软硬件卸载加速开发调试
            \item  熟悉Linux内核网络设备子系统
            \item  熟悉网卡相关的软硬件联调
        \end{itemize}
        \normalsize \par
    \datedsubsection{\textbf{英语方面}:英语六级、四级581}{}
\sepspace

%***********过往经历**************
\section{工作经历}
    \datedsubsection{\textit{软件开发工程师},研究院技术架构部,{\ruijie}}{2019.5 - 至今}
    \vspace{0.2em}

    %\Content
    %{工作职责:在研究院技术架构部,负责Linux下网卡驱动适配、开发和优化。网卡驱动涉及厂商intel, octeon, realtek, broadcom。}
    %{工作内容:网卡驱动与协议层的对接,网卡驱动的内部优化,网卡驱动的接口设计。}
    %{}
    \textbf{主要职责:} 
    负责网卡驱动代码模块的适配开发和维护。包括交换机CPU网卡驱动,智能网卡驱动。

    \projsssec{\textit{模块1:负责交换机平台CPU网卡驱动的适配开发}}\par

    % \begin{description}
    %     \item[工作简介:]
    %     在研究院技术架构部,对接交换机产品线,
    %     负责所有cpu网卡驱动适配开发,工程优化和维护。方便事业部进行网卡功能配置。
    %     网卡驱动涉及intel, octeon, realtek, broadcom。
    % \end{description}

    \hspace{2em}在研究院技术架构部,对接交换机产品线,
    负责所有CPU网卡驱动适配开发,工程优化和维护。方便事业部进行网卡功能配置。
    网卡驱动涉及Intel, Octeon, Realtek, Broadcom。工作内容大致如下:

    \begin{itemize}[parsep=0.5ex]
        \item 搭建网卡适配框架层,保证新品网卡的适配速度和质量。
        增加设备抽象层,统一各系列网卡的管理与收发包接口。
        增加代码的可扩展性,保证新品网卡的开发周期、质量和可维护性。
        \item 具体网卡驱动适配。将原生的网卡驱动适配至设备抽象层,包括收发包中与
        自研协议层的对接,功能管理接口适配,收发包流程优化,网卡驱动的接口设计。
        \item 网卡驱动的性能优化。驱动收发包流程中,减少报文的拷贝,增加多队列和skb导入、发包多队列均衡,尽量复用内核网络驱动框架。
        \item 网卡性能测试工具适配导入。导入Pktgen发包工具
        \item 框架的调试手段优化
        \item 用户态网卡驱动开发
    \end{itemize}

    \projsssec{\textit{模块2:负责智能网卡的驱动开发}}\par

    \begin{itemize}[parsep=0.5ex]
        \item 项目简介:智能网卡应用在云计算场景下,可将虚拟交换机的数据面功能卸载至其中,
              从而提升虚拟交换机的转发性能。
        \item 工作内容:在OVS-DPDK的高性能网络处理框架下,开发智能网卡驱动,涉及其基本功能管理,
              软硬件流表管理等。
        % \item 工作内容。在智能网卡预研项目中,基于OVS-DPDK的框架,负责智能网卡的驱动适配与开发。
        % 工作内容:智能网卡的基本功能管理、流表卸载管理与适配。
        % 基于DPDK/OVS实现软硬件卸载加速开发。
    \end{itemize}

    \sepspace

\section{项目经历}
    \datedsubsection{\textbf{锐捷网络}}{2021}

    \begin{itemize}[parsep=0.5ex]
        \item \textbf{国产化交换机}: 网卡驱动适配开发。涉及Realtek和Intel网卡。
        \item \textbf{内部虚拟交换机}: 网卡驱动适配开发。双协议栈兼容设计,内核协议栈与自研协议栈
        \item \textbf{高端路由器}: 发包测试工具。导入适配Pktgen来完成框架的流量测试。
        \item \textbf{国产化路由器}: 用户态网卡驱动开发。
        \item \textbf{智能网卡预研}: 基于DPDK/OVS实现软硬件卸载加速开发调试。
    \end{itemize}
	\normalsize \par

\section{论文}

\href{https://ieeexplore.ieee.org/abstract/document/9019181}{``Fault Tolerant Control for the Quadrotor Attitude Using a SOS Approach"} IEEE Chinese Guidance, Navigation and Control Conference, 2018

\end{document}
